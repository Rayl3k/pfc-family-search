\subsection{Descripció de la funcionalitat}

    \paragraph{}
    La funcionalitat evolució geogràfica d'un cognom permet explorar el nombre de persones nascudes, enregistrades a les bases de dades de FamilySearch, amb un cognom específic, per un conjunt de països seleccionats, en un any o interval de temps definit.

    Aquesta funcionalitat explota la funció de conveniència \emph{getResultsCount()}, inclosa en la resposta de cerca de persones del SDK de FamilySearch.

    Com ja s'ha comentat en diversos llocs de la memòria, els resultats d'una cerca de persones, retornen el nombre de persones indicades en el paràmetre \emph{count} o quinze per defecte. Com més persones es demanin, més tardarà el SDK a processar la petició.

    Tanmateix, independentment del nombre de persones demanades, el SDK disposa d’accés al nombre total de registres que compleixen les condicions de cerca i posa aquest valor a disposició dels desenvolupadors, mitjançant la funció de conveniència esmentada.

    D'aquesta forma, podem realitzar una cerca utilitzant el paràmetre \emph{count}, configurat a zero i obtenir així, el nombre total de persones que compleixen les condicions de cerca, mitjançant la funció de conveniència i sense sacrificar velocitat de resposta. Com que aquesta funcionalitat no pretén accedir als detalls de les persones en cap moment, no importa si aquestes no són retornades.

    La cerca permet configurar el cognom, països i any o anys, pels que s'ha de realitzar la cerca. A continuació llistem els paràmetres introduïbles per l'usuari:

    \begin{itemize}
        \item \textbf{Cognom:} Qualsevol valor introduït serà donat per vàlid, sempre i quan no es deixi el camp en blanc.
        \item \textbf{Països:} Per seleccionar un país, només cal marcar el checkbox situat a l’esquerra del seu nom. Els països estan agrupats per continents i dins de cada continent, els països es troben ordenats en ordre alfabètic.
        \item \textbf{Any de naixement:} Indica l'any pel qual es vol llençar la cerca o el primer any de l’interval que vol ser considerat. L'any ha de ser introduït en un format de quatre dígits. Per exemple, 2016.
        \item \textbf{Rang:} El \emph{rang} és un paràmetre opcional. En cas d’omplir-lo, indica quin és l'últim any de l’interval que es vol ser considerat. També ha de ser introduït en format de quatre dígits. Conjuntament, al paràmetre \emph{Any de naixement}, representa el període de temps comprés entre: \emph{Any de naixement--Rang}.
        \item \textbf{Interval:} El paràmetre \emph{interval}, només és necessari si s'ha especificat el camp \emph{Rang}. Aquest paràmetre, serveix per indicar, cada quants anys volem que es realitzi una fotografia, del nombre de persones nascudes amb un cognom determinat, en el conjunt de països seleccionats. Per exemple, si el paràmetre \emph{Any de naixement} és 1920, el paràmetre \emph{rang} és 1940 i \emph{l’interval} és 10, es realitzarà una foto de l'estat, pels anys: 1920, 1930 i 1940.
    \end{itemize}

    Dit això, cal tenir en compte, que per cada combinació d'any i país, es realitzarà una crida diferent, al SDK, demanant el nombre d'instàncies de persones nascudes amb el cognom indicat.

    S'utilitza aquesta aproximació, ja que l'alternativa consisteix a cercar el nombre de persones nascudes, amb el cognom seleccionat, pel rang de temps desitjat i navegar pels detalls de cada una de les persones de la resposta, per tal de geolocalitzar o cercar, el país de naixement d’aquesta. Aquesta aproximació, resulta bastant inviable si tenim en compte el volum de dades contingut a producció.

    Com a últim detall tècnic de la funcionalitat, comentar que s'obté la fotografia cada cert període de temps, en comptes d'agafar totes les instàncies de persones nascudes entre els diferents períodes, a causa d’unes discrepàncies en la forma de realitzar la cerca entre el SDK i la pàgina oficial de FamilySearch.

    Tot i que la documentació del SDK indica que permet la utilització de rangs de dates, en el paràmetre data de naixement, a la pràctica, no és així. La utilització d'aquesta opció, en el SDK, es tradueix en l’obtenció de resultats incorrectes.

    En utilitzar la funcionalitat, interval de dates, el SDK no retorna resultats. Després d'indagar una mica, es va descobrir que el motiu pel qual el SDK i la web oficial de FamilySearch, tenien comportaments dispars, era per la utilització d’un paràmetre de cerca diferent de cara a la introducció de dates.

    El SDK, realitza una crida a l'API de FamilySearch mitjançant el paràmetre especificat: \emph{birthDate}, mentre que la web FamilySearch utilitza el paràmetre \emph{birth\_year}. La diferència, resideix en què el paràmetre \emph{birthDate}, en realitat permet la inclusió del dia i mes de naixement en diferents formats, però no rangs de dates; mentre que el paràmetre \emph{birth\_year}, només considera els anys i per tant, si suporta la utilització d’intervals de temps.

    Així doncs, sembla que el paràmetre \emph{birth\_year}, no és suportat pel SDK de forma natural, ni tampoc funciona si aquest és forçat de forma manual des del codi. Per tots aquests motius, es va decidir que la funcionalitat implementada pintes la foto cada cert nombre d'anys, en comptes de recopilar la informació de tots els anys entre els diferents intervals.

    Retornant a la funcionalitat, un cop l’usuari llença una cerca, es mostra a l'usuari informació sobre la configuració de la cerca, la duració estimada i el percentatge completat fins al moment.

    A mesura que el SDK va retornant dades, es pinten diferents gràfics, que il·lustren, de formes diferents, el nombre d'instàncies del cognom introduït trobades a cada país.

    En concret s’han implementat tres tipus de gràfics diferents:

    \begin{itemize}
        \item \textbf{Mapa geogràfic:} Aquest mapa del globus terraqüi desplegat, emplena, amb diferents tonalitats de color verd, el nombre d’instàncies del cognom cercat pels diferents països. La foscor del color de cada país, indica la quantitat d'instàncies trobades respecte als altres països.
        \item \textbf{Gràfic de barres:} Aquest gràfic, ordena els països cercats, de més a menys instàncies trobades pel cognom seleccionat. Es mostra la mateixa informació que en el mapa geogràfic, però en format de gràfic de barres ordenat.
        \item \textbf{Gràfic de línies:} Només disponible quan la cerca conté dades per més d'un any. Mostra la quantitat d'instàncies trobades, a cada país, per cada any de l’interval. Permet observar l'evolució temporal de cada país respecte als altres.
    \end{itemize}

    Un cop finalitza la cerca de dades, es pot navegar pels gràfics dels diferents anys, mitjançant uns controls de navegació específics, situats al capdamunt de la zona de resultats.
