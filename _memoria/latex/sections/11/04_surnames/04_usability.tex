\subsection{Aspectes d'usabilitat considerats}

    \paragraph{}
    Aquesta funcionalitat aprofita molts conceptes d'usabilitat explicats en la funcionalitat de cerca a l’arbre familiar i que per tant, no repetirem en aquesta secció, però si anomenarem.

    \begin{itemize}
        \item La funcionalitat també implementa la interacció sobre certes capçaleres del formulari, per tal expandir o contraure parts d'aquest. També utilitza una barra sobreposada al final de la finestra, per facilitar l'accés al botó de cerca.
        \item Els missatges d'error que puguin aparèixer en prémer el botó de cerca o causats per la fallida del SDK, es representen de la mateixa forma que els apartats de la funcionalitat anterior.
        \item La navegació vertical animada, també forma part d'aquesta funcionalitat, quan l'usuari realitza interaccions que canvien la seva posició en la pàgina.
    \end{itemize}

    A continuació se citen alguns aspectes d'usabilitat propis d'aquesta funcionalitat.

    \subsubsection{Impressió dels gràfics}

\paragraph{}
Aquesta funcionalitat, disposa d’una barra de navegació, que permeten canviar l’any sobre el qual els gràfics mostren informació. Aquesta barra, es troba a l’inici de la zona de resultats i es manté fixa i sobreposada als altres elements del HTML, si s’arriba a certa profunditat en la pàgina web.

Aquesta barra de navegació, cobreix un doble objectiu. Primerament, informar a l'usuari sobre a quin any pertanyen les dades dels gràfics pintats i en segon lloc, permetre la navegació a través de les dades dels diferents anys, si aquestes existeixen.

La navegació a través dels gràfics de diferents anys, només es troba disponible, si s'han demanat dades per més d'un any. A més a més, s’han creat regles per impedir l'accés a gràfics que no existeixen, a través d’aquests controls.

Les anteriors imatges~\ref{fig:geomap}, \ref{fig:barchart} i \ref{fig:linechart}, mostren aquesta barra de navegació.

    \subsubsection{Progressió de la cerca}

\paragraph{}
Un dels aspectes d'usabilitat més interessants d'aquesta funcionalitat és la secció dedicada a mostrar la progressió de la cerca. Si ja havíem comentat que a mesura que es reben les dades dels diferents anys, aquestes són representades en els gràfics pertinents, aquesta secció de la pàgina compleix un rol similar, però més informatiu.

En el moment que es llança una cerca, l'usuari és transportat a aquesta secció. Durant el temps de cerca, aquesta secció mostra el cognom pel qual s'està realitzant la cerca, la duració estimada calculada mitjançant la fórmula: \emph{númeroPaïsos} $\times$ \emph{númeroAnys} $\times$ \emph{apiDELAY}, la informació sobre el país i any dels quals s'està esperant la resposta per part del SDK i una barra del progrés actual, respecta el total estimat.

Una cosa que cal tenir en compte és que les crides al SDK són asíncrones i que per tant, no està garantit que el retorn d'aquestes segueixi el mateix ordre que l'ordre d'enviament. Això significa, que potser, el bloc HTML que mostra la progressió de la cerca, pot indicar que s’estan esperant dades que ja han arribat o viceversa. De totes maneres, s’espera que en la gran majoria dels casos, hi hagi una correlació entre l’ordre d’enviament i retorn de les peticions al SDK.

En qualsevol cas, el fet més important relatiu a la barra de progrés, és que aquesta arribi al 100\% quan s'han processat totes les crides a l’API i que cada crida emmagatzemi el resultat a les caselles de la matriu que li pertoca. Recordem que això és aconseguit gràcies als paràmetres \emph{i} i \emph{k}, encapsulats en les crides.

Quan la cerca és completada, canvia l'estat dels components de la secció i s'indica que la cerca ha estat completada pel cognom especificat, s'elimina el temps estimat de finalització i és substituït per una indicació sobre la localització dels resultats i s'indica el nombre total de països i anys cercats. L'efecte de moviment en la barra de progressió, també és eliminat, per tal d'evitar confondre a l'usuari.

La imatge\ref{fig:waitingSurnames} mostra dos estats diferents de la secció progressió de la cerca.

\begin{figure}
    \includegraphics[width=\linewidth]{11/03_surnamesSearch/06_waitingDesktop}
    \includegraphics[width=\linewidth]{11/03_surnamesSearch/07_waitingDesktopComplete}
    \centering
    \caption{Exemples de la secció progressió de la cerca}\label{fig:waitingSurnames}
\end{figure}

    \subsubsection{Representació de l'estat}

\paragraph{}
Una de les característiques principals amb les quals s'ha intentat dotar l'aplicació és la capacitat de representar, en tot moment, l'estat actual de les funcionalitats independentment del que hagi passat anteriorment.

Les diferents micro transicions d'estat que succeeixen per aquesta funcionalitat i que poden no haver quedat cobertes en les seccions anteriors es llisten a continuació:

\begin{itemize}
    \item Quan es prem el botó de cerca, el text d'aquest canvia a `Searching now...' i passa a un estat de desactivació que n'impedeix la utilització fins que l'estat actual és resolt. Quan la cerca finalitza o un és produeix un error, l'estat del botó torna a la seva normalitat.
    \item Quan es realitza una nova cerca, els resultats de l'anterior s'amaguen per no causar confusió.
    \item Els missatges d'error provinents del SDK o errors de validació del formulari, desapareixen quan es llança una nova operació de cerca.
    \item Quan s'utilitzen les fletxes de navegació, de la barra de navegació, els gràfics apareixen i desapareixen per tal de fer palpable que aquests han estat refrescats.
    \item S'ha inclòs, dins de cada continent, la possibilitat de seleccionar o desseleccionar tots els checkboxes, mitjançant un botó.
    \item El gràfic de línies només es mostra si existeix més d'un any de dades, evitant d’aquesta forma la redundància amb el gràfic de barres, en el cas que només s’hagin consultat dades per un any concret.
    \item El gràfic de barres ordena els països de més instàncies a menys, per facilitar-ne la comprensió i obtenir una millora visual.
\end{itemize}

