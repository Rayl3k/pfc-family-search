\subsubsection{Representació de l'estat}

\paragraph{}
Una de les característiques principals amb les quals s'ha intentat dotar l'aplicació és la capacitat de representar, en tot moment, l'estat actual de les funcionalitats independentment del que hagi passat anteriorment.

Les diferents micro transicions d'estat que succeeixen per aquesta funcionalitat i que poden no haver quedat cobertes en les seccions anteriors es llisten a continuació:

\begin{itemize}
    \item Quan es prem el botó de cerca, el text d'aquest canvia a `Searching now...' i passa a un estat de desactivació que n'impedeix la utilització fins que l'estat actual és resolt. Quan la cerca finalitza o un és produeix un error, l'estat del botó torna a la seva normalitat.
    \item Quan es realitza una nova cerca, els resultats de l'anterior s'amaguen per no causar confusió.
    \item Els missatges d'error provinents del SDK o errors de validació del formulari, desapareixen quan es llança una nova operació de cerca.
    \item Quan s'utilitzen les fletxes de navegació, de la barra de navegació, els gràfics apareixen i desapareixen per tal de fer palpable que aquests han estat refrescats.
    \item S'ha inclòs, dins de cada continent, la possibilitat de seleccionar o desseleccionar tots els checkboxes, mitjançant un botó.
    \item El gràfic de línies només es mostra si existeix més d'un any de dades, evitant d’aquesta forma la redundància amb el gràfic de barres, en el cas que només s’hagin consultat dades per un any concret.
    \item El gràfic de barres ordena els països de més instàncies a menys, per facilitar-ne la comprensió i obtenir una millora visual.
\end{itemize}
