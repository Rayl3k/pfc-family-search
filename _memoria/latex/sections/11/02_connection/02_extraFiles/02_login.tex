\subsubsection{La funció d'identificació amb FamilySearch}

\paragraph{}
La funció d'identificació es realitza des de la pàgina \emph{login.html} de l'aplicació i el controlador que la gestiona està inclosa en el fitxer \emph{login.js}.

Quan l'usuari pressiona el botó `Access FamilySearch', mostrat a la figura~\ref{fig:fsLogin}, el controlador executa la funció \emph{getAccessToken()}, proporcionada pel SDK de FamilySearch. Aquesta funció, obre un pop-up que permet a l'usuari identificar-se i n'espera la resposta de forma asíncrona.

En cas que l'usuari s'identifiqui de forma correcta, s'emmagatzema el testTokenValue retornat pel SDK a l'espai local del navegador i és llença la funció \emph{serverLogIn(testTokenValue)}, que s'encarregarà de mantenir la concordança d'estat entre un token vàlid de FamilySearch i el nostre servidor.

\begin{lstlisting}[style=rawOwn,caption={Petició al SDK de Javascript d'un token d'identificació}]
client.getAccessToken().then(function(testTokenValue) {
    localStorage.token = testTokenValue;
    serverLogIn(testTokenValue);
});
\end{lstlisting}

La funció \emph{serverLogIn()} està inclosa en el fitxer \emph{client.js}, que recordem, està inclòs a totes les pàgines del web. Aquesta funció, realitza una crida AJAX al servidor passant com a paràmetre el valor \emph{testTokenValue} retornat per l'API. Un cop el servidor ha processat la petició, la funció definida en el paràmetre \emph{success}, redirigeix l'aplicació a la pàgina indicada pel servidor.

\begin{lstlisting}[style=rawOwn,caption={Crida AJAX al servidor amb el token retornat pel SDK}]
function serverLogIn(apiToken) {
    $.ajax({
        type: `POST',
        url: `/token/login',
        data: JSON.stringify({token: apiToken}),
        dataType: `json',
        contentType: `application/json; charset=UTF-8',
        success: function(data) { ... }
    });
}
\end{lstlisting}

El servidor, per la seva banda, en rebre la petició AJAX del client, crea una galeta de sessió segura i no modificable pel client amb el valor del paràmetre testTokenValue codificat. Mentre aquesta galeta existeixi, el servidor permetrà accedir a l'usuari a les pàgines d'exemples implementats.

\begin{lstlisting}[style=rawOwn,caption={Petició POST d'identificació processada pel servidor}]
app.post(`/token/login', function(req, res) {
    req.session.logged = req.session.logged || req.body.token;
    res.end(`{`redirect' : `/examples'}');
});
\end{lstlisting}
