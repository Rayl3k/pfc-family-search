\subsubsection{La funció de desconnexió amb FamilySearch}

\paragraph{}
Aquesta funció s'invoca de forma automàtica quan el `token' expira o quan l'usuari utilitza la funcionalitat de desconnexió situada a la barra de navegació.

La funció de desconnexió amb FamilySearch, realitza el procés invers realitzat per les funcions \emph{getAccessToken()} i \emph{serverLogIn()} descrites a l'apartat anterior. Aquesta, invalida el `token' configurat en la instància del client, elimina el valor contingut a l'espai local del navegador i sincronitza l'estat amb el servidor mitjançant una crida AJAX com la de l'apartat anterior.

\begin{lstlisting}[style=rawOwn,caption={Cridar AJAX al servidor per indicar que l'usuari s'ha desconnectat}]
function serverLogOut() {
    client.invalidateAccessToken();
    localStorage.removeItem(`token');
    $.ajax({
        type: `POST',
        url: `/token/logout',
        ...
    });
}
\end{lstlisting}

Per la seva banda, el servidor elimina la galeta de sessió, simplement assignant-li el valor \emph{null} i demana al controlador que redirigeixi a l'usuari cap a la pàgina principal de l'aplicació.

\begin{lstlisting}[style=rawOwn,caption={Petició POST de desconnexió processada pel servidor}]
app.post(`/token/logout', function(req, res) {
    req.session = null;
    res.end(`{`redirect' : `/'}');
});
\end{lstlisting}
