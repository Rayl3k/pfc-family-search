\subsubsection{Generació de taules dinàmiques amb informació sobre un recurs}

\paragraph{}
Com s'ha especificat en l'apartat anterior, la informació referent als diversos recursos disponibles dels detalls d'una persona, s'imprimeix en diferents taules.

La idea de com representar aquestes taules neix de l'aplicació d'exemple del SDK de Javascript. Mitjançant l'adaptació del codi d'aquest, per ajustar-lo a les necessitats del nostre projecte, s'aconsegueix, mitjançant la combinació de Bootstrap, javascript i jQuery, la generació de taules dinàmiques.

Es requereixen taules dinàmiques, ja que el nombre d'entrades en recursos com l'historial de canvis o el nombre de fills d'una parella, no és estàndard i per tant, s'ha d'adaptar per cada persona en concret.

La generació de les taules es realitza mitjançant la creació de matrius de dades, on cada cel·la és composta per un vector de dos valors. El primer, indica el tipus del camp de la cel·la, mentre que el segon, el contingut.

Mostrem en el següent bloc de codi, les dues primeres fileres de la matriu de dades relativa a la informació bàsica d'una persona.

\begin{lstlisting}[style=rawOwn,caption={Exemple de les dues primeres files, d'una matriu de dades}]
var displayProperties = createPanelTable(`Display information', [
    [
        [`th', `ID'],
        [`th', `Gender'],
        [`th', `Lifespan'],
        [`th', `Living']
    ],
    [
        [`td', person.getId()],
        [`td', person.getDisplayGender()],
        [`td', person.getDisplayLifeSpan()],
        [`td', person.isLiving()]
    ], ...
];
\end{lstlisting}

Un cop s'han creat les diferents matrius de dades, s'utilitza la funció \emph{craetePanelTable(header, rows)}, per transformar-les en un bloc de codi HTML que pinti la taules dins d'un panell de Bootstrap. Tot seguit, mostrem el codi que tradueix el bloc principal de files a HTML.

\begin{lstlisting}[style=rawOwn,caption={Transformació de files d'una matriu en HTML}]
for(var i = 0; i < rows.length; i++){
    var $row = $(`<tr>').appendTo($table);
    for(var j = 0; j < rows[i].length; j++){
        $(`<'+rows[i][j][0]+`>').text(rows[i][j][1]).appendTo($row);
    }
}
\end{lstlisting}
