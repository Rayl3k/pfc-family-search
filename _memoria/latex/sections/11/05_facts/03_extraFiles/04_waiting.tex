\subsubsection{Progressió de la cerca}

\paragraph{}
La progressió de la cerca és una secció del HTML molt similar al de la funcionalitat expansió geogràfica d’un cognom.

La finalitat d'aquesta secció és proporcionar a l'usuari un indicar del progrés de cerca realitzat fins al moment i el temps estimat per la finalització d'aquesta. L'objectiu, reduir la frustració de l'usuari i ensenyar de forma transparent, la feina ja realitzada.

No entrarem en els detalls més tècnics d'aquesta funcionalitat, pel fet que aquests ja han estat exposats en la funcionalitat anterior.

La informació que es mostra a l'usuari en aquesta funcionalitat és el tipus d'esdeveniment pel qual s'està realitzant la cerca, el país introduït i l'any central sobre el qual es realitza la cerca.

Aquesta funcionalitat, també mostra una barra de progrés que va completant-se a intervals de 9\%, a mesura que el SDK va resolent les diferents peticions enviades des del client.

Un cop ha finalitzat la cerca i el gràfic de línies ha estat pintat, el nou estat s'indica en aquesta secció, mitjançant un canvi en el text i l’eliminació de l'efecte de moviment en la barra de progrés.

La figura~\ref{fig:waitingFacts} mostra els dos estats diferents d'aquesta secció de la funcionalitat.

\begin{figure}
    \includegraphics[width=\linewidth]{11/04_factsSearcher/02_midSearch}
    \includegraphics[width=\linewidth]{11/04_factsSearcher/03_completeSearch}
    \centering
    \caption{Exemples de la secció progressió de la cerca}\label{fig:waitingFacts}
\end{figure}
