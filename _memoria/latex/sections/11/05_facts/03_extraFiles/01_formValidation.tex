\subsubsection{Validació del formulari de cerca}

\paragraph{}
Aquesta funcionalitat, utilitza el mateix sistema de validació en línia i validació en el moment de cerca, explicada en detall per la funcionalitat anterior, evolució geogràfica d’un cognom.

De la mateixa forma que les dues funcionalitats anteriors, aquesta també escapa els valors obtinguts del formulari, abans d’enviar-los al SDK, per tal d'evitar injeccions de codi i atacs al sistema.

Les regles de validació per cada un dels camps del formulari es llisten a continuació:

\begin{itemize}
    \item \textbf{Esdeveniment:} En principi no pot existir un estat en què cap dels esdeveniments es trobi seleccionat, però en cas de forcar-ho per edició del HTML, es dispara un error en el cas que cap tipus d'esdeveniment estigui seleccionat.
    \item \textbf{Localització:} Qualsevol localització és acceptada sempre i quan el camp no es deixi en blanc. Suggerim des d'aquesta part de la memòria, la utilització d’un país com a paràmetre de proves i que es tinguin presents les consideracions d'utilització descrites en l’apartat anterior.
    \item \textbf{Any central:} Aquest camp és acceptat com a vàlid, si el valor introduït té longitud quatre i és un número.
\end{itemize}

Per observar el resultat visual de la validació en línia o validació en el moment de cerca, es pot observar la figura~\ref{fig:surnamesError} de la funcionalitat anterior.
