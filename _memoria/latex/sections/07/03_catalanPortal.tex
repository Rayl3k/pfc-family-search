\section{Portal de cerca localitzat al Català}

    \paragraph{}
    En aquesta memòria ja hem parlat de la funcionalitat de localització habilitada per part de FamilySearch i encara que aquesta suporta la localització a la llengua espanyola, no ho fa per la catalana.

    La idea d'aquesta funcionalitat és oferir a l'usuari un portal de cerca en català sobre les dades de FamilySearch, que no només faciliti la comprensió de la cerca a aquelles persones que vulguin utilitzar el català, sinó que també localitzi, en la mesura que sigui possible, la resposta retornada per l'API.

    Per exemple, es podria localitzar la informació relativa a l'estat actual d'una persona: living o deceased, que podria ser mostrada com a viva o difunta. De la mateixa forma, es pot localitzar tot el contingut de la resposta, des del nom dels camps d'informació, al contingut d'aquests en algunes situacions.

    Alguns exemples de possibles extensions pel projecte són:

    \begin{itemize}
        \item Restringir la cerca del portal a només Catalunya, amb la possibilitat de desactivar la funció. A més a més, oferir ajuda de refinació a la cerca, de cara a introduir les diferents províncies o ciutats, evitant que l'usuari introdueixi valors invàlids.
        \item Posar-se en contacte amb l'organització FamilySearch per convertir la localització realitzada a la llengua Catalana, en una localització oficial acceptada pel sistema.
    \end{itemize}
