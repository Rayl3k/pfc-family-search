\section{FamilySearch i la segona guerra mundial: La llista de Schindler}

    \paragraph{}
    Aquesta proposta pretén realitzar un estudi sobre un dels col·lectius que es va veure més afectat durant la segona guerra mundial, els jueus.

    Aquesta proposta de projecte ofereix a l'estudiant parcejar els cognoms coneguts d'aquelles persones que van formar la llista de Schindler, i realitzar un estudi d'aquests sobre les dades de FamilySearch.

    El projecte pot intentar respondre preguntes com: Van disminuir en gran quantitat el nombre de registres amb els cognoms indicats? Van realitzar aquestes persones un procés d'emigració a diferents indrets del món? Quina mena de documents enregistrats han quedat d'aquestes persones?

    La proposta que ens ocupa se'm va acudir quan vaig descobrir, en les fases prèvies del projecte, el certificat d'emigració de Wladyslaw Szpilman, també conegut, com el pianista de Varsòvia. La imatge~\ref{fig:thePianist} mostra el registre físic trobat a FamilySearch.

    \begin{figure}[h]
        \includegraphics[width=\linewidth]{07/thePianist}
        \centering
        \caption{Registre d'emigració de \emph{Wladyslaw Szpilman}\label{fig:thePianist}}
    \end{figure}
