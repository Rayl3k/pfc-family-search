\section{Les col·leccions de dades de FamilySearch}

    \paragraph{}
    L'objectiu d'aquesta aplicació és llegir les diferents col·leccions o fonts de dades de FamilySearch i llistar-ne, de quines regions i sobre quins períodes de temps, contenen registres genealògics.

    L'objectiu, és oferir als usuaris un portal d'informació que permeti, mitjançant la introducció d'un país, període de temps o ambdues condicions, visualitzar les diferents col·leccions disponibles i la informació específica d'aquestes.

    Per exemple, donada la introducció del país Espanya i el període 1500--2000, la funcionalitat hauria de llistar totes les col·leccions que contenen dades que compleixen aquestes condicions, quants registres totals suposen aquestes, quants d'aquests estan accessibles a través de FamilySearch, quans pendents d'indexar, etcètera, etcètera.

    Aquesta funcionalitat pretén respondre a un dels problemes principals de l'API i és la falta d'informació sobre la informació emmagatzemada. D'aquesta forma, mitjançant un petit anàlisi previ, podríem esbrinar si la informació que desitgem és probable que existeixi o no, sense perdre el temps en l'exploració manual de registres.
