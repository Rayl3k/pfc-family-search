\section{Projectes d'indexació}

    \paragraph{}
    Tot i que aquesta proposta no pretén interactuar directament amb l'API de FamilySearch, volíem realitzar com a mínim una proposta que estigués relacionada amb el procés d'indexació.

    Existeixen dos processos d'indexació diferents, els que es realitzen sobre fitxers amb un format específic i els que es basen en la transcripció d'imatges a través del software de FamilySearch. Aquesta proposta de projecte, és en realitat dividida, en dues diferents.

    La primera, aconseguir accés a algun registre genealògic local o posar-se amb contacte amb alguna organització que vulgui pujar el contingut de registres amb un format específic, al núvol. Sobre aquest registre, implementar un sistema d'automatització que transcrigui les dades i les prepari per ser enviades a FamilySearch.

    La segona possibilitat, és realitzar un programa que interactuí amb les imatges digitalitzades, llegeixi les seccions de la imatge sobre les que s'ha d'extreure la informació i intenti informar a l'usuari, que no omplir automàticament, sobre el contingut dels camps.

    La gràcia d'aquest projecte és que es podria realitzar sense preocupar-nos per la certificació de l'aplicació, ja que per indexar registres un només s'ha de declarar com a voluntari i començar a experimentar.

    Aquestes dues propostes esdevenen, amb una alta probabilitat, bastant complexes, per aquest motiu, es prega a l'estudiant que realitzi un bon estudi previ sobre l'abast i viabilitat del que vol realitzar abans d'inscriure el projecte.
