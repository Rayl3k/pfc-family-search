\section{FamilySearch i la segona guerra mundial: Natalitat i Defuncions}

    \paragraph{}
    Durant el transcurs del temps han succeït un gran nombre d'esdeveniments que han afectat a la població mundial de diferents formes. Un dels conflictes que ha causat més repertori, ha estat la segona guerra mundial.

    L'objectiu d'aquesta proposta de projecte és observar l'impacte que va tenir aquest esdeveniment en l'índex de natalitat i defuncions, dels diferents països implicats, a través dels anys del conflicte. Es recomana ampliar la finestra de temps estudiat més enllà dels anys del conflicte per observar quins eren els valors normals, previs i posteriors, a l'esdeveniment.

    L'estudiant haurà de trobar la forma d'escalar les dades de cada país segons el volum de registres disponibles.

    Una altra tasca que pot realitzar l'estudiant, és comparar els valors obtinguts a través de FamilySearch amb les dades oficials de la segona guerra mundial i respondre preguntes de l'estil: Es corresponen els països amb un increment de defuncions més elevat amb els que van patir més durant la segona guerra mundial?
