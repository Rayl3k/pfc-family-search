\section{Introducció}

    \paragraph{}
    Aquesta secció recopila un conjunt d'idees que poden servidor com a projectes finals de carrera per estudiants de la Facultat d'Informàtica de Barcelona.

    L'objectiu de cada una de les propostes no és la de representar un enunciat tancat, sinó oferir pistes sobre diferents implementacions possibles que puguin inspirar als estudiants a modificar-les, combinar-les, reduir-les, ampliar-les o crear-ne de noves.

    Com es podrà veure, moltes d'aquestes propostes giren al voltant d'estudis històrics o la validació de les dades emmagatzemades en els sistemes de FamilySearch respecte a la realitat. El motiu, és que més enllà de la funcionalitat de cerca, la principal preocupació sobre aquestes dades és amb quin grau d'exactitud representen la realitat que les envolta.

    A més a més, recordar que l'API de FamilySearch es troba sempre en constant evolució, i que per tant, és una bona idea revisar la viabilitat de cada proposta abans de decidir, amb total certesa, el projecte que es vol realitzar.

    Així doncs, les propostes que s’ofereixen a continuació se centren bastant en la utilització del conjunt de dades que creiem més complet de cara a realitzar projectes finals de carrera amb ell. És per aquest motiu, que moltes de les propostes giren al voltant de les dades dels Estats Units, que recordem, representen el 62% del total de registres accessibles a través de l’API.

    La major part de propostes restants, se centren a respondre algunes preguntes específiques sobre l’API que aquest projecte no ha pogut estudiar. Aquestes, pretenen esbrinar el grau de fidelitat dels diferents blocs d’informació emmagatzemats per FamilySearch, respecte a la realitat coneguda.

    Finalment, algunes de les idees proposades giren al voltant de la creació de funcionalitats i per tant, l’èxit d’aquestes no depèn tant del conjunt de dades accessible.
