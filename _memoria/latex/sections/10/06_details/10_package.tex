\subsection{El fitxer package.json: Complements de l'aplicació}

    \paragraph{}
    El fitxer \emph{package.json}, s'utilitza per configurar l'aplicació web quan aquesta és desplegada al núvol. La part del codi més interessant és la que específica les dependències de l'aplicació i que dictamina els components que seran instal·lats, per tal de garantir el correcte funcionament d'aquesta.

    \begin{lstlisting}[style=rawOwn,caption={Dependències declarades en el fitxer pacakge.json}]
"dependencies": {
    "body-parser": "^1.15.2",
    "bootstrap": "3.3.6",
    "cookie-session": "2.0.0-alpha.1",
    "express": "4.14.0",
    "mustache-express": "1.2.2"
}
    \end{lstlisting}

    Els diferents paquets compleixen les següents funcionalitats:

    \begin{itemize}
        \item \textbf{Body-parser:} Utilitzat per capturar els paràmetres enviats des del frontal al servidor.
        \item \textbf{Bootstrap:} Com ja s'ha comentat, utilitzat per controlar l'estructura de les pàgines i la utilització de components genèrics.
        \item \textbf{Cookie-session:} Utilitzat per crear galetes de sessió no editables i signades, per tal d'evitar atacs a la seguretat dels usuaris.
        \item \textbf{Express:} Framework sobre el que es programarà l'aplicació Node.js.
        \item \textbf{Mustache-express:} Càrrega del llenguatge de plantilles pel framework Express.
    \end{itemize}
