\subsection{El fitxer header.html: Configuració de la pàgina}

    \paragraph{}
    Aquest fitxer s'encarrega de configurar la capçalera de les pàgines del nostre domini i obrir el cos del codi. També es declaren en aquesta secció els arxius CSS a carregar i les fonts a utilitzar.

    Les línies de codi més interessants són les que configuren el `viewport' del dispositiu, perquè aquest s'adapti de forma adequada a qualsevol pantalla i l'etiqueta necessària per indicar al navegador la codificació en què es troben els caràcters (línies 1 i 4 del següent codi).

    \begin{lstlisting}[style=rawOwn,caption={Capçaleres incloses en el fitxer header.html}]
<meta charset=`utf-8'>
<meta http-equiv=`X-UA-Compatible' content=`IE=edge'>
<title>PFC - Family Search API Study</title>
<meta name=`viewport' content=`width=device-width, initial-scale=1'>
<link href=`/node_modules/.../bootstrap.min.css' rel=`stylesheet'>
<link href=`/assets/css/style.css' rel=`stylesheet'>
<link href=`fontNumber1' rel=`stylesheet' type=`text/css'>
<link href=`fontNumber2' rel=`stylesheet' type=`text/css'>
    \end{lstlisting}
