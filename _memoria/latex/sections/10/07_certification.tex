\section{Certificant l'aplicació amb FamilySearch}

    \paragraph{}
    Com s'havia comentat en la quarta secció d'aquesta memòria, en la que s'introduïa l'API de FamilySearch, per tal d'obtenir accés a les dades de producció cal certificar les aplicacions.

    El procés de certificació, pot ser vist com una validació de l'aplicació per part de l'organització FamilySearch, per assegurar que no es realitza cap operació que pugui afectar al rendiment de l'API, la integritat de les dades o la seguretat del sistema.

    En el cas de la nostra aplicació, el conjunt de passos a realitzar per tal de certificar l'aplicació web s'exposen a continuació:

    \begin{enumerate}
        \item Aplicar l'aplicació per certificació des de la secció `les meves aplicacions' del portal de desenvolupadors de FamilySearch.
        \item Completar, signar i retornar l'acord d'afiliació de producte, les regles de seguretat i la petició d'obtenció d'una clau d'accés a producció.
        \item Registrar l'aplicació a FamilySearch i monitorar el procés de certificació.
        \item L'aplicació serà avaluada de diferents formes segons els diferents certificats que es vulguin obtenir. Aquests certificats van relacionats amb les operacions que realitza l'aplicació contra l'API.
    \end{enumerate}


    \subsection{Certificat d'autentificació}

    \paragraph{}
    El primer certificat que l'aplicació desenvolupada requereix és el relacionat amb el procés d'identificació dels usuaris a l'API de FamilySearch.

    La llista de consideracions a tenir en compte per obtenir el certificat són:

    \begin{itemize}
        \item Cada usuari ha d'adquirir un token d'identificació propi per tal de llegir dades de l'arbre familiar de FamilySearch.
        \item Els tokens d'identificació han de ser protegits. En cas que es vulgui emmagatzemar un token en una galeta del navegador, aquesta ha de ser una galeta segura.
        \item El tràfic és xifrat mitjançant el protocol SSL des de l'usuari fins a l'API de FamilySearch.
        \item L'autentificació de l'usuari és completada mitjançant la crida directa als protocols d'identificació OAuth 2 de FamilySearch. No es permet als tercers emmagatzemar informació sobre els noms d'usuari, contrasenyes i identificadors de sessió. En una aplicació web, l'autentificació pot ser realitzar, mitjançant la crida a un pop-up, al protocol d'autentificació de FamilySearch.
    \end{itemize}

    Donat que l'aplicació utilitza els serveis oferts pel SDK oficial de Javascript, de cara a l'autentificació, aquesta operació no hauria de presentar molts problemes de cara a obtenir el certificat.


    \subsection{Certificat de lectura}

    \paragraph{}
    El segon certificat que l'aplicació requereix va relacionat amb les operacions de lectura que realitza contra l'API de FamilySearch. Les principals restriccions a tenir en compte de cara a obtenir el certificat de lectura s'exposen a continuació:

    \begin{itemize}
        \item Procés d'autentificació certificat.
        \item Demostració de la correcta implementació de diferents funcions de lectura.
        \item Demostració de l'ús correcte de la cache de FamilySearch.
        \item Les aplicacions han de guiar a l'usuari a l'hora d'utilitzar les funcionalitats i ajudar-los a superar els possibles errors.
        \item Les aplicacions han de tenir en compte els estàndards del mercat a l'hora d'evitar atacs a la seguretat mitjançant injeccions de codi i altres vulnerabilitats.
        \item Complir amb les bones pràctiques de seguretat:
        \begin{itemize}
            \item Només està premés mostrar informació regulada de persones vives a les persones autentificades amb FamilySearch.
            \item La informació de persones difuntes no regulada pot ser mostrada a qualsevol usuari.
            \item Les dades locals emmagatzemades durant la sessió en el navegador, han de ser eliminades al final d'aquesta.
            \item Són permeses les tasques d'elevat temps de processat sempre i quan es compleixin certes regulacions.
            \item Les aplicacions poden guardar informació genealògica, de persones difuntes, obtinguda a través d'usuaris identificats amb FamilySearch.
            \item Les aplicacions poden guardar, però no fer pública, informació genealògica de persones vives.
            \item Les aplicacions poden emmagatzemar els identificadors de les persones, amb dades regulades, però no les dades regulades per se.
            \item Les aplicacions no poden emmagatzemar relacions concretes que indiquin que una persona va trobar-se, en un lloc concret, en una data concreta.
        \end{itemize}
    \end{itemize}

    Un altre cop, pel fet d'utilitzar el SDK oficial de Javascript, molts dels punts principals necessaris per obtenir la certificació de lectura haurien d'estar coberts. Un exemple clar, és el de la correcta implementació de les funcions de lectura. Donat que aquestes no són realment controlades per nosaltres, no hi hauria d'haver cap problema.

    Pel que fa a les bones pràctiques de seguretat, s'han utilitzat precaucions per evitar problemes com la injecció de codi a través dels formularis de l'aplicació i la nostra aplicació no emmagatzema cap mena d'informació de l'usuari, ni de les dades retornades, més enllà del token d'identificació.


    \subsection{Procés de certificació}
