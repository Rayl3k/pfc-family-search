\section{Accés a l'aplicació web i codi de l'aplicació}

    \paragraph{}
    L’aplicació web es troba desplegada al núvol sota l'URL:

    \begin{displayquote}
        https://pfc-family-search.herokuapp.com/
    \end{displayquote}

    Per accedir a la zona específica d’exemples, la que s’encarrega de mostrar els diferents exemples d’interacció amb l’API, fa falta utilitzar el següent usuari i contrasenya:

    \begin{itemize}
        \item \textbf{Usuari:} tum000145207
        \item \textbf{Contrasenya:} 1234pass
    \end{itemize}

    Per altra banda, el codi de les aplicacions web sol ser extens en nombre de línies i mostrar-lo en aquesta memòria resulta impossible. El codi realitzat ocupa un total de [nombre de línies] repartides un [x\%] en HTML, un [y\%] en Javascript i jQuery i un [z\%] en css.

    Tot el codi de l’aplicació pot ser trobat en el repositori GitHub accessible a través del següent URL:

    \begin{displayquote}
        https://github.com/sinh15/pfc-family-search
    \end{displayquote}

    L’estructura del codi serà presentada més endavant, en aquesta mateixa secció de la memòria, però principalment, el servidor està compost pel fitxer \emph{app.js}, els fitxers HTML es troben a la carpeta \emph{views} i els fitxers Javascript i jQuery, a la carpeta \emph{assets}.
