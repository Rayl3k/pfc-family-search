\section{Hosting de l'aplicació}

    \paragraph{}
    Com ja s'ha indicat en el primer apartat d'aquesta secció de la memòria, l'aplicació es troba accessible a través del domini:

    \begin{displayquote}
        https://pfc-family-search.herokuapp.com/
    \end{displayquote}

    Hem escollit la plataforma Heroku per desplegar l'aplicació, ja que oferia un ampli ventall d'eines i documentació, que resultaven ideals per un desenvolupador novici de la plataforma Node.js.

    No obstant això, no van ser només aquestes facilitats i eines les que ens van portar a desplegar l'aplicació a Heroku, sinó també el fet que el seu pla de hosting gratuït s'ajustava en gran mesura al que el projecte requeria i en cas d'acabar necessitant més, sempre existia la possibilitat d'augmentar la capacitat de processat necessària.

    Les característiques que més ens van atreure de la plataforma Heroku es presenten d'una en una en els següents apartats.

    \subsection{Fàcil configuració}

    \paragraph{}
    Per una aplicació simple com la desenvolupada per aquest projecte, no fa falta canviar res en el codi de la nostra aplicació per tal de desplegar-la a Heroku. Només hem d'incloure un nou fitxer, a la carpeta arrel del projecte i amb el nom \emph{Procfile}, que indica el tipus d'aplicació i la comanda que ha ser utilitzada per tal d'iniciar-la.

    En el nostre cas, el fitxer \emph{Procfile} conte la següent línea de codi:

    \begin{displayquote}
        > web: node app.js
    \end{displayquote}


    \subsection{Fàcil desplegament al núvol}

    \paragraph{}
    La plataforma Heroku es troba molt ben integrada amb Github, l'eina que hem utilitzat per mantenir sincronitzats els desenvolupaments en les diferents estacions de treball. Gràcies a aquesta integració, Heroku disposa d'una comanda que ens permet afegir un remot, a la carpeta del nostre projecte.

    \begin{displayquote}
        > heroku git:remote -a pfc-family-search
    \end{displayquote}

    Un cop tenim el remot de Heroku configurat, de la mateixa forma que podem enviar el codi de la nostra estació de treball al núvol, el podem enviar a Heroku per tal de desplegar la nostra aplicació web. Això és tan fàcil com llençar la següent instrucció.

    \begin{displayquote}
        > git push heroku master
    \end{displayquote}


    \subsection{Entorn de proves local}

    \paragraph{}
    A part de l'entorn de proves local que podem configurar mitjançant la instal·lació de Node.js en el nostre sistema, Heroku també ens permet simular un entorn de producció Heroku, en l'àmbit local. S'aconsegueix mitjançant la següent instrucció:

    \begin{displayquote}
        > heroku local web
    \end{displayquote}

    Tot i que no aporta gaires diferències respecte a desplegar l'aplicació en local mitjançant Node.js de la forma convencional, pot esdevenir útil sota certes circumstàncies.


    \subsection{Decent versió gratuita}

    \paragraph{}
    Durant tot el procés de desenvolupament, l'aplicació es trobava sota el paquet de funcionalitats gratuït ofert per Heroku.

    Aquest ofereix les següents funcionalitats:

    \begin{itemize}
        \item Desplegament des de repositoris GIT.
        \item Actualitzacions automàtiques.
        \item Auto reparació d'aplicacions.
        \item Logs del sistema.
        \item Nombre de processos diferents suportats: 2
        \item 1000 hores mensuals de \emph{dyno} actius. L'aplicació s'adorm després de 30 minuts d'inactivitat.
        \item Dominis personalitzables.
        \item 512MB de RAM
    \end{itemize}


    \subsection{Fàcil escalatge de l'aplicació}

    \paragraph{}
    En cas que es desitgi millorar la capacitat de concurrència de l'aplicació, per poder rebre i processar més peticions en paral·lel i realitzar més processos diferents al mateix temps, aquesta és fàcilment escalable mitjançant la inclusió de nous \emph{dynos}.

    Els \emph{dynos} són els contenidors que executen les comandes dels usuaris. Per la nostra aplicació web, bàsicament es necessiten \emph{dynos} que processin el tràfic HTTP i HTTPS. Gràcies al fet que el nostre servidor és bastant simple (hem posat la lògica d'interacció amb l'API de FamilySearch a la capa del controlador) és probable que no faci falta augmentar el nombre de \emph{dynos} inicial.

    De totes maneres, aquest és fàcilment escalable mitjançant una simple comanda, sempre i quan el nostre pla contractat amb Heroku no sigui el gratuït. Si volem augmentar, per exemple, a 2 el nombre de \emph{dynos} disponibles, executaríem la següent comanda:

    \begin{displayquote}
        > heroku ps:scale web=2
    \end{displayquote}
