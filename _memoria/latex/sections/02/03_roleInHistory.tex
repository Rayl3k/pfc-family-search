\section{El paper de la genealogia en el transcurs de la història}

    \paragraph{}
    Com s’ha comentat en l'apartat anterior, avui en dia la genealogia és una ciència que busca en gran mesura respondre preguntes de caràcter personal, no obstant això, aquest no va ser sempre el seu objectiu principal.

    Històricament, en les cultures occidentals, les persones estaven interessades a mantenir-se ben informades sobre la seva ascendència de cara a fer latents les seves connexions amb nobles i governants. Generalment, la intenció era protegir la seva situació privilegiada o escapar de la precarietat. En aquesta època, el terme genealogia compartia significat amb el d'\gls{heraldica}, terme usat avui en dia per la ciència que estudia els escuts d’armes. Així doncs, fins a finals del segle XIX, la genealogia deixava la seva marca en la història com a eina utilitzada principalment per aquells amb drets de poder o riquesa adquirits a través de l’herència.

    Aquest exemple, que bé ens podria semblar distant en el temps, no és l’única mostra dels impactes històrics relacionats amb aquesta ciència i com veurem a continuació, existeixen altres exemples molt més propers.

    No cal tornar gaires anys enredera per veure com durant l’època de l’alemanya nazi ser capaç de demostrar l’afiliació a la "raça suprema" era necessari per sobreviure o inclòs poder casar-se de forma legal. Per aquest motiu, no ens ha d’estranyar que avui en dia, Alemanya, segueixi sense fer públics la major part dels registres genealògics del segle XX, doncs els fets històrics han portat a percebre la història familiar com un atac, o amenaça, a la privacitat i seguretat de les persones. Les conseqüències d'aquesta època de la història són conegudes per tothom i un no pot evitar entreveure certes relacions amb el camp de la genealogia.

    Per situar un exemple que ens ocupi si pot ser encara més de ple, podem veure el valor de la memòria històrica i dels sentiments d’unió amb els nostres avantpassats arran de la gran quantitat de publicacions i missatges personals, recordant als seus avantpassats i als temps que els va tocar viure, en relació al vuitantè aniversari de l'esclat de la guerra civil espanyola. De fet, Catalunya és un altre clar exemple contemporani, conjuntament amb Alemanya, de com la memòria històrica pot ser present en la cultura, vida i sentiments de bona part d’una nació. Tant en l’àmbit personal, com col·lectiu.

    Així doncs, podem concloure que la genealogia, no tant com a ciència sinó com eina, va desenvolupar, desenvolupa i probablement, seguirà desenvolupant, un paper important en la història de la humanitat. No hem d’oblidar que els problemes racials segueixen molt presents en l'actualitat de les nostres societats, Estats Units, n'ha estat últimament un clar exemple, i que és la raça sinó una característica més de les nostres característiques de naixement, o en altres paraules, de les nostres dades genealògiques.
