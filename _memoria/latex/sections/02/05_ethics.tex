\section{Els codis ètics en la genealogia}

    \paragraph{}
    El fet que els genealogistes tinguin accés i treballin amb informació pública, però simultàniament, personal, provoca que la professió es vegi envoltada de codis ètics que tractin de protegir la integritat d'aquesta, el sentiment de professionalitat i la moralitat d'aquells que interactuen amb les dades.

    Els codis morals giren al voltant de dos eixos. La protecció de la informació referent a les persones vives i les bones praxis de cara a la manipulació i tractament de les dades.

    El primer eix, tal com hem indicat, fa referència a protegir a aquelles persones que encara són vives. En concret, es tracta d'evitar, en la mesura que sigui possible, publicar informació de caràcter personal que poguí resultar compromesa per un individu en concret.

    El codi moral que hi ha en el rerefons és que a cap persona li agradaria trobar informació personal publicada al núvol, on tothom la pot accedir, pel simple fet que es tracta d'informació `pública'. Per tant, cal respectar la privacitat de les persones i no publicar fets o dades compromeses independentment de l'opinió personal del genealogista.

    Com ja s'ha mencionat amb anterioritat, la genealogia tracta en gran mesura d'estudiar els nostres avantpassats i la història de tots aquells que van existir abans que nosaltres, per tant, el xafardeig i la tafaneria no tenen cabuda dins dels codis ètics i morals de la professió.

    La segona branca ètica es correspon a un seguit de bones praxis de cara a la utilització d'informació genealògica. Tot i que no es tracta d'un manual oficial, la següent llista de `regles' serveix per descriure i fer-nos una idea amb un alt grau de fiabilitat, del que significa el concepte de bones praxis en aquesta ciència:

    \begin{itemize}
        \item El genealogista mantindrà les fonts de referència i les citarà quan utilitzi dades fetes públiques per un altre individual o col·lectiu.
        \item El genealogista no compartirà ni utilitzarà informació no contrastada o amb altes probabilitats de ser errònia.
        \item El genealogista transmetrà seguretat i confiança a aquells que facin ús dels seus serveis.
        \item El genealogista donarà suport a aquelles iniciatives que preservin els fitxers públics i l'accés a aquests.
        \item Es tractarà amb cordialitat i respecte al personal de les facilitats d'investigació.
        \item El genealogista ajudarà en la mesura que sigui possible als altres genealogistes i organitzacions dedicades a la genealogia.
        \item El genealogista compartirà els resultats dels seus estudis i investigacions.
        \item No està permesa la invenció ni exageració de la informació.
        \item El genealogista complirà amb les lleis en rigor dels conjunts de dades que utilitzarà en els seus estudis.
    \end{itemize}
