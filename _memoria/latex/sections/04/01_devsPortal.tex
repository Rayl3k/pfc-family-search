\section{El portal de desenvolupadors}

    \paragraph{}
    Tota la informació disponible per tal de poder començar a familiaritzar-se amb l’\gls{API} de FamilySearch, pot ser trobada en el portal de desenvolupadors.

    Aquest apartat de la web està format per diferents seccions, malauradament, l’estructura no acaba de resultar del tot clara per una persona que vulgui iniciar-se per primer cop en l'ús d'aquesta \gls{API}.

    Si ens enfoquem més en la documentació disponible, que no pas en l'estructura proposada per l’organització, podem veure que la informació es podria distribuir, en certa forma, en els següents grups:

    \begin{itemize}
        \item \textbf{Requisits tècnics:} Conjunt d’informació necessària per comprendre l’estructura de l'\gls{API}, els formats de dades que maneja i els passos necessaris per començar a interactuar amb aquesta.
        \item \textbf{Recursos disponibles i rutes d’accés:} Informació detallada sobre cada recurs accessible a través de l'\gls{API}. En concret, disposa dels detalls de com accedir al recurs, les operacions que es poden realitzar sobre ell, la informació que conté i quines són les connexions amb altres recursos.
        \item \textbf{Evolució i canvis produïts a l'\gls{API}:} Informació semi ordenada de com l'\gls{API} s’ha vist evolucionada al llarg del temps i un recull dels canvis produïts sobre els recursos, procés de certificació, material de documentació i eines de desenvolupament.
        \item \textbf{Serveis extres oferts per l'\gls{API}:} Aquest recull d’articles conceptualitza característiques de l'\gls{API} com poden ser els recursos d'\emph{emmagatzematge}, \emph{localització} o \emph{throttling}.
        \item \textbf{Eines de desenvolupament:} Recull d’entorns de desenvolupament i eines extres que poden facilitar la feina del desenvolupador.
        \item \textbf{Certificació:} Recull la informació necessària per gestionar els diferents processos de certificació i informació sobre les regulacions a les quals s'ha de fer front en cas de voler certificar l'aplicació.
    \end{itemize}
