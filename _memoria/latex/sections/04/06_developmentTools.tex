\section{Eines de desenvolupament}

    \paragraph{}
    FamilySearch posa a disposició dels desenvolupadors un seguit d'eines destinades a facilitar les seves implementacions contra l'API.

    Són dos els recursos principals que permeten apropar la utilització de l'API a qualsevol desenvolupador. Els SDKs i les aplicacions d'exemple.

    \subsection{Els SDK de FamilySearch}

        \paragraph{}
        Els \gls{SDK}, són un conjunt d'eines que faciliten la creació d'aplicacions per un software, hardware, framework, sistema de computació, videojoc, sistema operatiu o plataforma de desenvolupament determinada.

        En el cas de FamilySearch, es tracta d'eines de desenvolupament destinades a un llenguatge de programació específic.

        Però, perquè esdevé útil la utilització d'un SDK? En el cas de FamilySearch un SDK ens ofereix la possibilitat de comunicar-nos amb l'API sense haver de preocupar-nos, de forma directa, en com les URIs han de ser codificades o en parcejar els JSON/XML de les respostes. Taques que poden esdevenir tedioses i repetitives.

        Per tant, els SDK permeten als desenvolupadors centrar-se en el desenvolupament de funcionalitats i no en la gestió de comunicacions amb l'API.

        FamilySearch disposa en l'actualitat de sis SDKs diferents. Tots ells faciliten la interacció amb l'API, encara que no tots permeten les mateixes funcionalitats. A més a més, alguns d'aquests SDK són oficials, significant que són desenvolupats i mantinguts per l'organització FamilySearch, mentre que altres, són creats i gestionats per la comunitat.

        Els diferents SDK disponibles es llisten a continuació:

        \begin{itemize}
            \item \textbf{Java SDK:} El Java SDK és un SDK oficial que serveix per crear aplicacions d'escriptori amb el llenguatge Java.
            \item \textbf{PHP SDK:} El SDK de PHP és un SDK oficial per crear aplicacions web. Desafortunadament, no es troba del tot actualitzat i l'última versió d'aquest no integra masses de les funcionalitats que es poden realitzar contra l'API.
            \item \textbf{C Sharp SDK:} El SDK de C Sharp és un altre dels SDK oficials i serveix per crear aplicacions amb la tecnologia .NET.
            \item \textbf{Javascript SDK:} El SDK de Javascript és l'últim SDK oficial i està orientat sobretot a la creació d'aplicacions web. És el SDK que s'ha utilitzat en aquest projecte ja que cobreix gran part de les funcionalitats de l'API i en facilita l'ús.
            \item \textbf{Python SDK:} El SDK de Python resultaria molt atractiu de cara a tota mena d'aplicacions, però per desgràcia, encara es troba en fase de desenvolupament i sense dates concretes de finalització.
            \item \textbf{Ruby SDK:} La perla de Ruby és l'últim SDK disponible per FamilySearch i es tracta d'un SDK no oficial.
        \end{itemize}


    \subsection{Les aplicacions d'exemple}

        \paragraph{}
        Les aplicacions d'exemple consisteixen en aplicacions de codi obert que permeten entendre, de forma més pràctica, com funcionen els SDK.

        Solen cobrir un conjunt d'operacions bàsiques, com poden ser, per exemple, llegir l'usuari connectat, realitzar una cerca bàsica o la lectura d'una persona concreta de l'arbre familiar. En definitiva, representen un bon punt de partida per aquells que vulguin familiaritzar-se amb un SDK concret i comprendre'n les bases que els permetin l'elaboració de funcionalitats més complexes en el futur.

        De la mateixa forma que els SDK, FamilySearch disposa de sis aplicacions d'exemple, aquest cop, sent només oficials les aplicacions que utilitzen els SDK de Javascript i Java.


    \subsection{Altres eines intressants}

        \paragraph{}
        L'últim recurs per desenvolupadors que FamilySearch ofereix, és un llistat d'eines que poden ser interessants o esdevenir útils, de cara a la realització de proves amb l'API.

        Entre aquestes, destaquen diferents intermediaris que serveixen per simular peticions HTTP i HTTPS contra l'API de FamilySearch i una utilitat, no oficial, que en teoria permet copiar dades de producció als entorns de desenvolupament.
