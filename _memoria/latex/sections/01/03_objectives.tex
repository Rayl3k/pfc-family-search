\section{Objectius}

    \paragraph{}
    Tot i la incertesa de quins seran els detalls finals del projecte en el moment de començar, l’abast dels objectius principals sí que es troba ben marcat i definit. Com hem comentat, els objectius principals del projecte giren al voltant de tres grans blocs:

    \begin{itemize}
        \item Estudi de l’\gls{API} de FamilySearch
        \begin{itemize}
            \item Estudi de les peces d’informació accessibles i utilitzables.
            \item Relacions entre les diferents peces d’informació.
            \item Fer transparent, als futurs estudiants, el potencial real de les dades emmagatzemades, els principals obstacles a tenir en compte si es vol utilitzar aquesta API i la viabilitat d’utilitzar FamilySearch a l’hora de realitzar un projecte final de carrera.
        \end{itemize}
        \item Bateria d'idees relacionades amb l’\gls{API} de FamilySearch tenint en compte les restriccions del sistema i la informació disponible per servir com a futurs projectes o com a font d'inspiració.
        \item Avaluació i implementació d’exemples que es comuniquin o nodreixin de l’\gls{API} de FamilySearch, n’exposin la informació disponible i ofereixin una idea bàsica de les oportunitats i complicacions que aquesta comporta. Aquest objectiu es divideix en diversos apartats:
        \begin{itemize}
            \item Estudiar les diferents opcions disponibles per la implementació.
            \item Escollir el tipus d’aplicació a desenvolupar i estudiar les tecnologies necessàries per desenvolupar l'aplicació.
            \item Definició de l’abast i esquelet de l’aplicació, així com dels exemples que seran desenvolupats i definició dels requisits funcionals i no funcionals de l'aplicació.
            \item Implementació de l’aplicació.
            \item Procés de certificació i proves del sistema.
        \end{itemize}
    \end{itemize}

    \paragraph{}
    Aquest projecte també presenta un seguit d’objectius secundaris o més aviat, objectius personals, que en certa forma m’agradaria deixar plasmats en la memòria.

    \begin{itemize}
        \item Adquisició del coneixement necessari sobre el funcionament d'una pàgina web, quins són els seus components principals i com interactuen.
        \item Estudi d'algunes de les tecnologies més usades en el mercat actualment.
        \item Aplicació bàsica dels coneixements d’usabilitat i experiència d’usuari adquirits durant l’etapa d’interí.
        \item Aprenentatge del llenguatge de maquetació de text LaTeX per tal de formatar articles i documents de caràcter tècnic.
    \end{itemize}
