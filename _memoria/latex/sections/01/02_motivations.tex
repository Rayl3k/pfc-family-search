\section{Motivació i context}

    \paragraph{}
    Durant el transcurs de la carrera són moltes i diverses les vessants de la informàtica que ens van ser introduïdes. D’aquestes, sempre vaig sentir més afinitat per aquelles que requerien allunyar-se un pèl dels detalls més tècnics i s'enfocaven en un exercici d’abstracció i conceptualització centrat en la comprensió d’un problema quotidià, el disseny d’una solució i finalment, la seva execució.

    Entenc doncs, que donada la situació, no és d'estranyar que sentís una empatia més elevada per camps com la mineria de dades o l’enginyeria del software que no pas l'algorítmica o els sistemes operatius. Al mateix temps, sempre m’he considerat una persona dispersa a qui li agrada conèixer una mica de molts temes diferents i per tot això, buscava un projecte que em permetés trencar en certa forma amb aquests aspectes més tècnics de la informàtica i explorar un camp desconegut al qual es poguessin aplicar els coneixements adquirits durant la carrera.

    El projecte proposat per l’Enric complia doncs, en bona mesura, amb tot allò que jo buscava. La genealogia representava un camp que desconeixia per complet i el projecte en si era un full de ruta obert. Només la realització de la primera part d'aquest ens permetria comprendre com de profundes i completes podrien ser les propostes que el projecte originaria o quins serien els exemples a implementar.

    En aquest aspecte, el projecte resultava especialment atractiu, doncs la riquesa final d'aquest vindria donada per la capacitat de traduir els objectius d’un camp d’estudi com la genealogia, mitjançant el grup de dades disponible a través de FamilySearch, en propostes que fossin capaces de satisfer preguntes o inquietuds latents en la societat. A la vegada, aquest mateix aspecte convertia el projecte en aterridor, doncs no seria fins ben entrat en aquest, que comprendríem les opcions disponibles i fins a quin nivell podríem aprofundir en l'elaboració de propostes i exemples.

    Un últim aspecte que em va ajudar a decidir-me per aquest projecte va ser la influència de la feina d'interí que estava realitzava en aquell moment com a dissenyador d’experiència d’usuari. El projecte m’oferia la possibilitat d’implementar una pàgina web com a apartat tècnic i aquesta em permetia posar a prova els coneixements d’usabilitat i experiència d’usuari adquirits durant els darrers mesos. De la mateixa forma, s'obria la porta a desenvolupar les habilitats necessàries per programar una pàgina web, una part de la informàtica que no havia explorat durant la carrera i em feia certa gràcia.
