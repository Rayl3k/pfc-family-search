\section{Planificació}

    \paragraph{}
    L'objectiu d'aquest apartat de la memòria és presentar les diferents planificacions que s'han portat a terme per encarar el projecte en cada una de les convocatòries matriculades.

    \subsection{Planificació Febrer del 2014 -\ Juliol 2014}

        \paragraph{}
        Aquest projecte va ser matriculat per primera vegada al Febrer del 2014 amb la intenció de presentar-lo com a principis de juliol del mateix any. La idea inicial era aprofitar el mes de gener per avançar feina i disposar així d’un total de sis mesos per realitzar el projecte.

        Al març del 2014 vaig començar a treballar a jornada completa i el projecte va deixar d’avançar a la velocitat esperada. Es van començar a patir forts endarreriments sobre la planificació original fins al punt que el projecte va quedar completament aturat. La figura~\ref{fig:firstPlan} mostra en línies generals la planificació que s’hagués volgut portar a terme en cas de normalitat i ressaltat en vermell la part que es va veure interrompuda.

        \begin{figure}[h]
                \includegraphics[width=\linewidth]{01/firstPlan}
                \centering
                \caption{Planificació original \emph{Febrer 2014 -\ Juliol 2014}.\label{fig:firstPlan}}
        \end{figure}

        La falta de temps per realitzar un projecte acceptable, conjuntament, a la poca capacitat de maniobra de les que es va disposar entre els mesos de Març i Juliol, va provocar que es descartés la possibilitat de presentar el projecte durant la convocatòria prevista inicialment.

    \subsection{Planificació Febrer del 2016 -\ Setembre 2016}

        \paragraph{}
        A  conseqüència de l’extinció del pla d’enginyeries 2003, el projecte es torna a matricular al Febrer del 2016, tenint en consideració que s'hauria de començar pràcticament de 0.

        Donada la diferència de temps entre la primera inscripció i la segona, l’\gls{API} de FamilySearch s'havia vist sotmesa a grans canvis i la major part del material estudiat i coneixements tècnics adquirits fa dos anys, quedaven completament antiquats.

        A pesar de matricular el projecte a mitjans de Febrer es coneixia que aquest no podria ser començat amb agilitat fins a principis d’abril a causa d'una situació excepcional en l'àmbit laboral. Gràcies a la disponibilitat d’una pròrroga extraordinària, que permetia estendre el període d'entrega fins a finals de setembre, la finestra de temps disponible per completar el projecte rondava els cinc o sis mesos.

        Cal tenir en compte que la disponibilitat horària en el dia a dia de cara a treballar en el projecte era molt reduïda i en conseqüència, realitzar una bona planificació era essencial si es volien evitar els mateixos problemes que van provocar l’abandonament del projecte en el seu primer intent.

        Tant en la figura~\ref{fig:actualPlan}, com en les seccions que segueixen a continuació, expliquem com es va ser planificada i executada la feina entre els mesos d'abril i setembre.

        \begin{figure}
            \includegraphics[scale=0.5, angle=90]{01/actualPlan}
            \centering
            \caption{Planificació final \emph{Febrer 2016 -\ Setembre 2016}.\label{fig:actualPlan}}
        \end{figure}

        \subsubsection{Segona quinzena de Març}

            \paragraph{}
            En aquest petit període de temps es va realitzar el primer estudi superficial sobre l'API de FamilySearch amb la finalitat d’observar quins canvis s'havien produït durant els darrers dos anys i com aquests podien afectar o modificar la proposta inicial inscrita del projecte.

            L'objectiu d'aquesta repassada ràpida era la de proporcionar una visió global sobre certes limitacions que podrien afectar el desenvolupament del projecte i ens permetés elaborar una planificació coherent de com afrontar i estructurar la feina a realitzar.

        \subsubsection{Primera quinzena d'Abril}

            \paragraph{}
            Tot i que l’estudi sobre la informació disponible a través de l’API es trobava en els seus inicis es va aprofitar aquesta quinzena per decidir quina mena d’aplicació volíem implementar. Aquesta decisió obriria pas a la recerca i estudi sobre quines tecnologies serien més adients de cara a la implementació dels exemples i les comunicacions amb l'API de FamilySearch.

            També s'aprofitaria aquesta quinzena per familiaritzant-nos amb la diferent documentació disponible sobre l'API de FamilySearch i plantejar-ne l’ordre d'estudi.

        \subsubsection{Segona quinzena d'Abril -\ Finals de Maig}

            \paragraph{}
            
