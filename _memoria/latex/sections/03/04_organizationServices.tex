\section{Serveis per organitzacions amb arxius genealògics}

    \paragraph{}
    Com ja s’ha comentat en seccions anteriors, FamilySearch no és només una organització dedicada a posar a disposició del públic registres genealògics, sinó que també pretenen ajudar a altres organitzacions a digitalitzar i publicar els seus documents al núvol de forma econòmica, ja sigui mitjançant la plataforma FamilySearch o la construcció de noves eines pròpies al núvol.

    Sigui com sigui, FamilySearch ofereix cinc serveis a altres organitzacions genealògiques:


    \subsection{Captura d'imatges}

    \paragraph{}
    Obtenir imatges de qualitat és normalment el procés més costos per aquelles organitzacions que volen digitalitzar els seus registres, tant en el sentit econòmic, com en base als recursos humans necessaris.

    El microfilm, que fins fa poc era l’estàndard en la indústria, comença a cedir pas al món digital i tant si l’objectiu de les organitzacions és digitalitzar el seu contingut mitjançant medis propis o utilitzant la tecnologia disponible en els centres de recerca de FamilySearch, aquests ofereixen la seva ajuda a les organitzacions que la sol·licitin.

    La tecnologia dCamX, utilitzada per FamilySearch, es caracteritza per la creació d'imatges d’alta qualitat, de forma eficaç, al mateix temps que es capturen meta dades de la imatge. Aquest aspecte, conjuntament amb un procés de publicació posterior fàcil i ràpid, fan que aquesta tecnologia estigui cridada a ser el nou estàndard a la indústria.


    \subsection{Conversió de formats digital}

    \paragraph{}
    Aquest servei està pensat per aquelles empreses o organitzacions que ja disposen d’una elevada quantitat de material en format de microfilm.

    La tecnologia digital ha canviat dràsticament com els registres són capturats, emmagatzemats i fets accessibles. Aquesta tecnologia segueix progressant i per tant resulta indispensable començar a adaptar-se al més aviat possible.

    FamilySearch posa a disposició de les organitzacions genealògiques la possibilitat d’utilitzar els mateixos processos i software que utilitzen ells per digitalitzar la seva col·lecció de més de 2,4 milions de microfilms. FamilySearch, també ofereix la possibilitat d’emmagatzemar els fitxers digitals d’aquestes organitzacions en els seus servidors, un cop convertits, si així ho prefereixen.


    \subsection{Indexació en línea}

    \paragraph{}
    Un cop un registre ha estat digitalitzat en forma d’imatge, la informació principal necessita ser extreta i transcrita per tal de poder produir índexs sobre els quals clients o usuaris puguin cercar.

    L’aplicació d’indexació en línea, creada per FamilySearch, permet, mitjançant la cadena de voluntaris, crear índexs de forma ràpida i precisa. Els arxius de les organitzacions, que així ho sol·licitin, podran disposar d’accés a aquesta cadena de voluntaris per digitalitzar els seus índexs o accés a les eines d’indexació auxiliars  que permeten la creació de projectes propis.


    \subsection{Accés en línea}

    \paragraph{}
    Si un document o registre no esdevé fàcilment accessible, resulta de poc valor pels usuaris. FamilySearch ofereix dos serveis diferents depenent de si les organitzacions desitgen fer públic l'accés a les seves dades a través de FamilySearch.org o no.

    En cas de voler per públics els registres, FamilySearch s’ofereix a penjar i mantenir els registres de forma econòmica. En cas de voler mantenir els registres en un àmbit privat, l’organització posa a disposició dels interessats les eines i experiència necessàries per crear un espai propi al núvol.


    \subsection{Preservació dels registres i fitxers físics}

    \paragraph{}
    FamilySearch ofereix l’opció a les organitzacions de custodiar còpies de seguretat dels seus fitxers, en el baül de tecnologia punta situat a las Granite Mountains. En l'actualitat, còpies d’arxius de microfilm i digitals, provinents de més de cent països diferents, es troben guardades en aquest baül per precaució.

    El fet de disposar de còpies de seguretat pels fitxers genealògics, suposa la salvació de registres en cas de terratrèmols, incendis, inundacions, tornados, guerres i actes humans no controlables en les seus oficials dels arxius.

    Les mesures de seguretat que FamilySearch utilitza pels seus registres i que a la vegada, queden a disposició d’altres organitzacions, són:

    \begin{itemize}
        \item Processos complets i automatitzats de comprovació, validació i actualització dels registres per garantir la màxima protecció possible.
        \item Migració gradual i eficient de registres cap a noves tecnologies quan els formats previs quedin obsolets, garantint així la seva accessibilitat a llarg termini.
        \item Processos de conversió i preservació de dades que compleixen amb les regulacions sobre els \gls{OAIS}.
        \item Col·leccions d’informació emmagatzemada al núvol i distribuïdes en diferents clústers arreu del món per garantir una alta capacitat d’emmagatzematge, escalabilitat i protecció contra els desastres.
        \item Utilització de les últimes tecnologies en el tractament de dades d'alta densitat.
        \item Procés ràpid i eficient de cara a processar l'arribada de nous registres al sistema. El sistema actual és capaç de processar més de vint terabytes al dia, mesura que incrementa, al mateix temps que la tecnologia avança.
        \item Servidors configurats en clústers virtuals per garantir una escalabilitat infinita.
        \item Facilitat amb control climàtic, prevenció de focs, fonts d’energia auxiliars per casos d’emergència i replicació d’arxius digitals.
    \end{itemize}


    \subsection{Conclusions sobre els serveis professionals}

    \paragraph{}
    En els apartats anteriors s'ha pogut observar com FamilySearch està clarament interessada a posar les seves tecnologies a disposició d'altres organitzacions genealògiques.

    Aquesta estratègia de col·laboració els permet incorporar a les seves bases de dades registres d'informació, d'altre forma inaccessibles i garantir la persistència de les dades davant d'esdeveniments no controlables, d'informació gestionada per altres organitzacions.
