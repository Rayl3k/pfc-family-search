\section{La història de FamilySearch}

    \paragraph{}
    FamilySearch, en els seus orígens coneguda com la Societat Genealògica de Utah, va néixer de les mans de l'església de Jesucrist dels Sants dels Darrers Dies l'any 1894. Aquesta església també és coneguda avui en dia pel nom de l'església mormona.

    L’organització porta acumulats a les seves espatlles més de cent anys de recerca i preservació d’arxius històrics i genealògics. Durant aquest recorregut, FamilySearch s’ha associat amb més de 10.000 arxius i 200.000 voluntaris de tota mena d’indrets.

    Mitjançant l’esforç col·lectiu d’aquestes organitzacions s’ha aconseguit preservar tant índexs com imatges d’arxiu de gran qualitat i posar tots aquests recursos a la disposició de milions de persones de forma gratuïta.

    Durant aquest trajecte de més de 100 anys iniciat l'any 1894, l'organització ha tingut l'oportunitat de celebrar grans èxits. Entre ells, destaquen els següents:

    \begin{itemize}
        \item \textbf{1938, Microfilm:} La societat genealògica de Utah és pionera en començar a utilitzar microfilm per filmar i emmagatzemar informació relativa a arxius genealògics arreu del món.
        \item \textbf{1942, Family group record archive:} Es crea, de forma manual, una indexació de les genealogies compartides fins aquell moment.
        \item \textbf{1963, Baül d’arxius a les Granite Mountains:} Es completa la creació d’un baül d’arxius amb tecnologia punta situat a les muntanyes pròximes a Salt Lake City, Utah, indret on resideix, avui en dia, la seu de l’organització. Es tracta d’una instal·lació climatitzada que ha estat utilitzada des de la seva creació per preservar còpies de microfilm i arxius digitals de més de cent països diferents.
        \item \textbf{1970, Primers centres d'història familiar:} S'introdueixen els primers centres d'història familiar. Aquests centres formen part d'una ramificació de llibreries que ofereixen accés gratuït a la informació continguda per més de 2,4 milions d'arxius en microfilm. Com hem esmentat en l'apartat anterior, avui en dia existeixen 4.765 d'aquests centres.
        \item \textbf{1985, L’estàndard GEDCOM:} En aquest any FamilySearch introdueix l'estàndard de \gls{GEDCOM}. L'estàndard GEDCOM consisteix en un conjunt d'especificacions i regles sobra com s'ha d'estructurar la informació genealògica de cara a compartir-la amb facilitat a través del núvol.
        \item \textbf{1995, Arbres genealògics digitalitzats:} Es dóna l’oportunitat als genealogistes de digitalitzar els arbres genealògics a la seva disposició i habilitar-ne l’accés a altres usuaris.
        \item \textbf{1998, Digitalització d’imatges:} FamilySearch comença a utilitzar tecnologies d'imatge digital per tal de capturar noves fonts de dades i transformar els milions de continguts, emmagatzemats fins ara en microfilm, en imatges digitals. La tecnologia també permet crear índexs de fàcil utilització que relacionen persones amb els continguts digitals.
        \item \textbf{1999, Nova pàgina web:} La pàgina web FamilySearch.org arriba al núvol. En la seva fase inicial, aquesta oferia la possibilitat de cercar informació en els registres històrics de forma relativament simple.
        \item \textbf{2012, Noves tecnologies digitals:} S’incorpora a les tecnologies utilitzades per FamilySearch la tecnologia dCamX, utilitzada per la digitalització de documents i la creació de  sales de lectura digital, responsables de facilitar les tasques de comunicació i alliberació de coneixement entre els diferents centres.
    \end{itemize}

    Un dels altres grans èxits de  l’organització, del que malauradament no es coneix la data exacte de creació, va ser \textbf{l’obertura de la FamilySearch \gls{API}.}

    Aquesta va suposar que aplicacions externes poguessin connectar-se a les bases de dades de FamilySearch i utilitzar-ne la informació d'una forma regulada i eficient. Resulta prou evident que sense l'existència d'aquesta \gls{API}, aquest projecte mai hagués pogut tenir lloc.
