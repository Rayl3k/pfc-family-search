\section{L'església mormona i la família}

    \paragraph{}
    Ja que el principal benefactor de FamilySearch és l'església mormona, creiem que és interessant estudiar de forma breu els orígens d'aquesta.

    L’església de Jesucrist dels Sants dels Darrers Dies és una església que considera que la religió hauria de tornar als seus inicis apostòlics. Va ser fundada pel nord-americà Joseph Smith el 6 d’abril del 1830, a l'oest de Nova York.

    Considerada actualment com la quarta comunitat cristiana més gran als Estats Units, troba situada la seva seu en l'actualitat a Salt Lake City, Utah. No obstant això, durant els inicis de l’església, Smith tenia la intenció de crear la Nova Jerusalem a prop de Nova York, en una ciutat que anomenaria \emph{Zion}.

    L’església, amb origen als voltants de Nova York, es va desplaçar cap a Kirtland, Ohio, des d'on va començar a expandir-se per Jackson County, Missouri, terra en què Smith volia situar la seu del col·lectiu en un futur pròxim.

    Joseph va veure contrariats els seus plans, quan l'any 1833, els colons van expulsar brutalment al col·lectiu de Missouri. Com que no disposaven dels recursos militars necessaris per recuperar el territori per la força, es van veure obligats a anar desplaçant-se al llarg de diferents localitzacions, sempre per culpa de conflictes amb els natius de les terres, fins a establir-se a Nauvoo, Illinois.

    Després de la mort de Smith, per tal d'evitar els conflictes armats amb els residents d'Illinois, el col·lectiu es va desplaçar cap a Nebraska i més endavant, durant l'any 1847, a les terres que serien conegudes com a Utah.

    Durant aquesta època, l'església es va veure sotmesa a grans pressions i crítiques a causa de la seva tolerància per la poligàmia. Les tensions entre el col·lectiu i el govern d'Estats Units anirien en augment, fins que l'any 1890, el congrés va disgregar l'església i es va apoderar de molts dels seus béns.

    Arribats aquest punt, l'església fundada per Smith, va decidir deixar de donar suport als matrimonis plurals, però sense desfer les famílies que ja es trobaven unides sota aquestes condicions.

    Durant el segle XX, l'església va créixer substancialment i es va veure sotmesa a un procés d'internacionalització, en gran mesura, gràcies a la feina dels missioners enviats a diferents indrets del món.

    Durant aquest període el col·lectiu es va convertir en un ferm defensor de les famílies nuclears, és a dir, de les famílies que consisteixen en dos progenitors i la seva descendència. L'església també va oposar-se en aquesta època a l'esmena pels drets igualitaris entre homes i dones, els casaments entre persones del mateix sexe i l'eutanàsia.

    Hem volgut redactar aquests paràgrafs previs sobre els orígens de l'església mormònica per tal de poder presentar, amb cert rigor històric, com l'església va veure canviat i evolucionat el concepte de família al llarg del temps.

    També queda latent, d'aquesta forma, com la història del col·lectiu es va veure marcada pel rebuig i el desterrament de moltes terres, fins al punt que van haver de recórrer a l'ús de missioners, repartits arreu del món, per tal de sobreviure com a religió.

    Així doncs, creiem que per aquest projecte no esdevé necessari entrar en més detall pel que fa a les doctrines i pràctiques de l'església, ni  enumerar quines són les principals diferencies entre l'església mormona i les altres corrents del cristianisme. Per altra banda, sí que volem realitzar una reflexió final sobre la posició actual de l'església mormona respecte a la família.

    Pels mormons, les famílies representen els lligams que uneixen a les persones en relacions personals i les connecten tant amb les passades com amb les futures generacions. Creuen que cap èxit en la vida, pot compensar el fracàs en l'àmbit familiar.

    Segons el seu punt de vista, construir nuclis familiars units i forts és el remei a molts dels fracassos que tenim les persones com a societat i creuen que la família inspira a l'individu a pensar més enllà de l'interès propi o la gratificació immediata i l'anima a entregar-se per altres persones, comunitats i a déu.

    Forma part també de la cultura mormona la pràctica o deure d'acumular i preservar, tant les històries dels seus avantpassats com les pròpies, en benefici d'aquells que encara estan per arribar, enllaçant, d'aquesta forma, generacions desconnectades d'una altra forma.

    Entenen la naturalesa real de la família, com un algú que transcendeix l'aquí i l'ara i que permet a les persones extreure forces d'aquells que van viure abans que nosaltres.

    Concloïen, si ajuntem les dues variables que van marcar l'esdevenir de l'església mormona fins als temps contemporanis, és a dir, la seva semi forçada internacionalització i la importància del nucli familiar en la seva cultura, no hauríem de mostrar-nos sorpresos pel fet que el col·lectiu s'hagi convertit en un dels referents mundials en el camp de la genealogia.
