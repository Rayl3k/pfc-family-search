\subsection{El recurs Canvi (Change)}

    \paragraph{}
    Un altre recurs transversal, per alguns dels recursos més importants de l'API de FamilySearch, és el recurs Canvi.

    Quan un usuari realitza alguna modificació de qualsevol mena, ja sigui sobre la informació d'una persona o sobre una relació de parella o parental, aquest canvi queda enregistrat per diversos motius.

    El primer, poder veure com les dades s'han anat modificat al llarg del temps i veure'n la progressió. El segon, poder recuperar un estat anterior en cas d'error o problema en el sistema.

    El recurs Canvi està format pels paràmetres mostrats a la taula~\ref{res:change}.

    \begin{center}
             \csvreader[
                separator=comma,
                before table=\sffamily\small,
                longtable={p{2cm-2\tabcolsep}p{3.5cm-2\tabcolsep}p{8.5cm-2\tabcolsep}},
                table head={\caption{Paràmetres del recurs Canvi}\label{res:change}\\\toprule%
                    \headentry{m{2cm-2\tabcolsep}}{Paràmetre}
                    & \headentry{m{3.4cm-2\tabcolsep}}{Format de Dades}
                    & \headentry{m{8.5cm-2\tabcolsep}}{Descripció}\\\midrule},
                late after line=\\\midrule,
                late after last line=\\\bottomrule,
             ]
             {./tables/05/06_others/change.csv}
             {param=\param,format=\format,desc=\desc}
             {\param&\format&\desc}
     \end{center}


     \subsubsection{L'enumeració changeObjectModifier}

     \paragraph{}
     L'enumeració changeObjectModifier segueix l'estructura de definició GEDCOMX. Com a tal, els valors possibles per l'enumeració segueixen la pauta:\\\verb|http://gedcomx.org/ + `changeObjectModifier'|

     La taula~\ref{enum:changeObjectModifier} mostra els possible valors de l'enumeració changeOperation.

     \begin{center}
         \csvreader[
            no head,
            separator=comma,
            table head={\caption{Valors possibles per l'enumeració changeObjectModifier}\label{enum:changeObjectModifier}},
            before table=\sffamily\small,
            longtable={|p{3cm}|p{3cm}|p{6cm}|},
            column count=4,
            late after head=\\\hline,
            late after line=\\\hline,
            late after last line=\\\hline,
         ]
         {./tables/05/06_others/changeObjectModifier.csv}
         {1=\one,2=\two,3=\three}
         {\one&\two&\three}
     \end{center}


     \subsubsection{L'enumeració changeOperation}

     \paragraph{}
     L'enumeració changeOperation segueix l'estructura de definició GEDCOMX. Com a tal, els valors possibles per l'enumeració segueixen la pauta:\\\verb|http://gedcomx.org/ + `changeOperation'|

     La taula~\ref{enum:changeOperation} mostra els possibles valors de l'enumeració changeOperation.

     \begin{center}
         \csvreader[
            no head,
            separator=comma,
            table head={\caption{Valors possibles per l'enumeració changeOperation}\label{enum:changeOperation}},
            before table=\sffamily\small,
            longtable={|p{3cm}|p{3cm}|p{3cm}|p{3cm}|},
            column count=4,
            late after head=\\\hline,
            late after line=\\\hline,
            late after last line=\\\hline,
         ]
         {./tables/05/06_others/changeOperation.csv}
         {1=\one,2=\two,3=\three,4=\four}
         {\one&\two&\three&\four}
     \end{center}


     \subsubsection{L'enumeració changeObjectType}

     \paragraph{}
     L'enumeració changeObjectType segueix l'estructura de definició GEDCOMX. Com a tal, els valors possibles per l'enumeració segueixen la pauta:\\\verb|http://gedcomx.org/ + `changeObjectType'|

     La taula~\ref{enum:changeObjectType} mostra els possibles valors de l'enumeració changeObjectType.

     \begin{center}
         \csvreader[
            no head,
            separator=comma,
            table head={\caption{Valors possibles per l'enumeració changeObjectType}\label{enum:changeObjectType}},
            before table=\sffamily\small,
            longtable={|p{3cm}|p{3cm}|p{3cm}|p{3cm}|},
            column count=4,
            late after head=\\\hline,
            late after line=\\\hline,
            late after last line=\\\hline,
         ]
         {./tables/05/06_others/changeObjectType.csv}
         {1=\one,2=\two,3=\three,4=\four}
         {\one&\two&\three&\four}
     \end{center}
