\subsection{El recurs Font de Dades (SourceDescription)}

    \paragraph{}
    Aquest recurs defineix una font de dades i n'emmagatzema tota la informació que la caracteritza. Les fonts de dades formen part d'una col·lecció i es troben relacionades a ella mitjançant enllaços hypermedia.

    La millor forma d'explicar aquest recurs és comprendre'n els paràmetres i aquests s'exposen a la taula~\ref{res:sourceDescription}. El recurs Font de Dades també hereta els paràmetres dels recursos Enllaços Hypermedia i Dades Extensibles que poden ser trobats a l'apartat `Altres recursos interessants'.

    \begin{center}
             \csvreader[
                separator=comma,
                before table=\sffamily\small,
                longtable={p{2cm-2\tabcolsep}p{3.5cm-2\tabcolsep}p{8.5cm-2\tabcolsep}},
                table head={\caption{Paràmetres del recurs Font de Dades}\label{res:sourceDescription}\\\toprule%
                    \headentry{m{2cm-2\tabcolsep}}{Paràmetre}
                    & \headentry{m{3.4cm-2\tabcolsep}}{Format de Dades}
                    & \headentry{m{8.5cm-2\tabcolsep}}{Descripció}\\\midrule},
                late after line=\\\midrule,
                late after last line=\\\bottomrule,
             ]
             {./tables/05/05_sources/sourceDescription.csv}
             {param=\param,format=\format,desc=\desc}
             {\param&\format&\desc}
     \end{center}


    \subsubsection{L'enumeració resourceType}

    L'enumeració resourceType segueix l'estructura de definició GEDCOMX. Com a tal, els valors possibles per l'enumeració segueixen la pauta:\\\verb|http://gedcomx.org/ + `resourceType'|

    La següent taula mostra els possibles valors per l'enumeració resourceType.

    \begin{center}
        \csvreader[
           no head,
           separator=comma,
           table head={\caption{Valors possibles per l'enumeració resourceType}\label{enum:resourceType}},
           before table=\sffamily\small,
           longtable={|p{3cm}|p{3cm}|p{3cm}|},
           column count=3,
           late after head=\\\hline,
           late after line=\\\hline,
           late after last line=\\\hline,
        ]
        {./tables/05/05_sources/resourceType.csv}
        {1=\one,2=\two,3=\three}
        {\one&\two&\three}
    \end{center}
