\subsection{El recurs Relació (Relationship)}

    \paragraph{}
    Aquest recurs s'utilitza en l'actualitat per representar només les relacions de parella. En el passat, també va ser utilitzat per representar relacions entre pares i fills, mitjançant l'ús del paràmetre \emph{type} i l'enumeració \emph{relationshipType}. Tanmateix, aquest ha caigut en el desús des de la incorporació del recurs Relacions Pares i Fill.

    Aquest recurs emmagatzema informació sobre les persones que conformen la relació i els esdeveniments relacionats a aquesta. Les dades pròpies del recurs es mostren a la taula~\ref{res:relationship} i també hereta les dels recursos Subjecte, Conclusió, Enllaços Hypermedia i Dades Extensibles que poden ser trobats a l'apartat `Altres recursos interessants'.

    \begin{center}
             \csvreader[
                separator=comma,
                before table=\sffamily\small,
                longtable={p{2cm-2\tabcolsep}p{3.5cm-2\tabcolsep}p{8.5cm-2\tabcolsep}},
                table head={\caption{Paràmetres del recurs Relació}\label{res:relationship}\\\toprule%
                    \headentry{m{2cm-2\tabcolsep}}{Paràmetre}
                    & \headentry{m{3.4cm-2\tabcolsep}}{Format de Dades}
                    & \headentry{m{8.5cm-2\tabcolsep}}{Descripció}\\\midrule},
                late after line=\\\midrule,
                late after last line=\\\bottomrule,
             ]
             {./tables/05/02_relationships/relation.csv}
             {param=\param,format=\format,desc=\desc}
             {\param&\format&\desc}
     \end{center}

     \clearpage


    \subsubsection{L'enumeració relationshipType}

    \paragraph{}
    L'enumeració relationshipType segueix l'estructura de definició GEDCOMX. Com a tal, els valors possibles per l'enumeració segueixen la pauta:\\\verb|http://gedcomx.org/ + `relationshipType'|

    La següent taula mostra els possibles valors per l'enumeració relationshipType, però com ja hem comentat, l'ús del valor ParentChild, ja no s'utilitza en favor del nou recurs Relació Pares i Fill.

    \begin{center}
        \csvreader[
           no head,
           separator=comma,
           table head={\caption{Valors possibles per l'enumeració relationshipType}\label{rel:relationshipType}},
           before table=\sffamily\small,
           longtable={|p{3cm}|p{3cm}|},
           column count=4,
           late after head=\\\hline,
           late after line=\\\hline,
           late after last line=\\\hline,
        ]
        {./tables/05/02_relationships/relationshipType.csv}
        {1=\one,2=\two}
        {\one&\two}
    \end{center}
