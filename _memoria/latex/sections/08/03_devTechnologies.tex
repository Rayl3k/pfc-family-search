\section{Tecnologies específiques pel desenvolupament de l'aplicació}

    \paragraph{}
    Apart de les tecnologies que permeten la creació i funcionament de l'aplicació web, hi ha un altre conjunt de tecnologies o eines que s'han utilitzat per facilitar el desenvolupament d'aquesta.

    \subsection{La tecnologia GitHub}

    A causa de les complicacions resultants de compaginar una jornada laboral a temps complet amb l'elaboració del projecte, el desenvolupament d’aquest s’ha realitzat des de diverses localitzacions diferents. En concret, han estat tres els principals ordinadors o localitzacions des de les que s’ha treballat: Casa, Portàtil i oficina de l'empresa.

    Per tal de facilitar la sincronització d'arxius entre les diferents estacions de treball, es va decidir utilitzar la tecnologia GitHub.

    GitHub és un repositori Git hostejat al núvol. Ofereix totes les funcionalitats pròpies de Git, destinades al control de revisions distribuïdes i control del codi font, afegint-ne algunes de noves.

    Un cop creat el repositori on tot el projecte queda emmagatzemat, els servidors de GitHub s’encarreguen de què aquest sigui disponible des de tot arreu. D’aquesta forma, cada estació de treball diferent pot descarregar-se el codi per primera vegada, mitjançant la instrucció:

    \begin{displayquote}
        git clone https://github.com/sinh15/pfc-family-search.git pfc-family-search
    \end{displayquote}

    És important evitar treballar en dues estacions de treball diferents, a menys que entre sessions de treball s’actualitzin els canvis en el repositori emmagatzemat al núvol, per tal d’evitar superposicions i conflictes entre les versions dels arxius.

    GitHub disposa d'eines per fer front a aquestes situacions, ja que en realitat és una eina pensada per treballar en equip sobre els mateixos arxius de codi. No obstant això, com que aquest projecte ha estat realitzat per una sola persona, s'ha prescindit de la utilització de branques i tots els canvis s'han aplicat sobre la branca mestra del projecte.

    Un cop s'acaba una sessió de feina, independentment de l'estació de treball, es poden pujar els canvis al núvol mitjançant tres simples instruccions:

    \begin{displayquote}
        git add .\\
        git commit -m ``sessió de treball X finalitzada''\\
        git push origin master
    \end{displayquote}

    De la mateixa forma, abans de començar a treballar des de qualsevol estació de treball, podem recuperar l'últim estat del projecte mitjançant la instrucció:

    \begin{displayquote}
        git pull origin master
    \end{displayquote}

    A part dels beneficis que aquesta eina ens proporciona de cara a treballar de forma distribuïda i en diferents entorns, serveix al mateix temps, de `backup' o reserva, en cas que alguna de les oficines de treball quedi malmesa o es vulgui recuperar una versió antiga d'algun fitxer del codi.


    \subsection{Node.js i Express en l'àmbit local}

    \paragraph{}
    Durant el desenvolupament, per tal de facilitar les proves sobre l’aplicació i no haver de realitzar un desplegament al núvol per cada canvi que es vulgui comprovar, aquesta s'ha programat en un entorn local.

    Aquest fet implica que el nostre sistema operatiu feia de servidor per l'aplicació web i aquesta només resultava accessible a través de la URL `http://localhost:8080'.

    Per tant, el nostre sistema local necessitava ser capaç d'emular les tecnologies que formarien part del servidor. A recordar, Node.js i Express.

    No entrarem a descriure la tecnologia Node.js ni Express, ja que ja ho hem fet en apartats anteriors. El que sí que volíem fer, era exposar que aquestes tecnologies havien estat instal·lades en l’àmbit local.


    \subsection{Tecnologia NPM}

    \paragraph{}
    Com s'ha indicat en un apartat anterior, NPM és un contenidor de paquets orientats a la plataforma Node.js. Aquesta tecnologia pot ser instal·lada en l'àmbit local i d'aquesta forma, descarregar extensions per les aplicacions Node.js que volem provar.

    Per instal·larar un conjunt de paquets, podem fer-ho introduint la instrucció: npm install, que s’encarrega d’instal·lar en l’àmbit local, totes aquelles dependències que hagin estat declarades en el servidor de l’aplicació web, o introduir la instrucció: npm install `package name’, per instal·lar un paquet en concret.


    \subsection{Paquet Nodemon}

    \paragraph{}
    Un paquet que volem destacar, instal·lat a través del repositori NPM, però que només ha estat utilitzat en l’entorn de desenvolupament local i no desplegat al núvol, s’anomena Nodemon.

    Nodemon observa els canvis realitzats en els fitxers de codi de la nostra aplicació i en cas que algun canvií, aquest reinicia el servidor que fa córrer l'aplicació de forma automàtica, per tant, sense necessitat de reiniciar-lo de forma manual i poder veure així els canvis a l'entorn local de forma immediata.

    Pot semblar un paquet que aporta poca funcionalitat, però quan et trobes implementat una pàgina web, generalment realitzes molts petits canvis en el codi dels quals vols observar-ne l'afecte abans de continuar programant. Per fer-ho, faria falta reiniciar manualment el servidor, però aquest paquet, ens estalvia aquesta tasca.

    El fet de ser un paquet que ha estalviat bastant de temps a l'hora de realitzar petites proves i observar-ne els canvis, volíem que trobes la seva representació en la memòria.
